%%%%%%%%%%%%%%%%%%%%%%%%%%%%%%%%%%%%%%%%%
% Stylish Article
% LaTeX Template
% Version 2.1 (1/10/15)
%
% This template has been downloaded from:
% http://www.LaTeXTemplates.com
%
% Original author:
% Mathias Legrand (legrand.mathias@gmail.com) 
% With extensive modifications by:
% Vel (vel@latextemplates.com)
%
% License:
% CC BY-NC-SA 3.0 (http://creativecommons.org/licenses/by-nc-sa/3.0/)
%1
%%%%%%%%%%%%%%%%%%%%%%%%%%%%%%%%%%%%%%%%%

%--------------------------------------------------------------# Path to your oh-my-zsh installation.--------------------------
%	PACKAGES AND OTHER DOCUMENT CONFIGURATIONS
%----------------------------------------------------------------------------------------

\documentclass[fleqn,10pt]{SelfArx} % Document font size and equations flushed left

\usepackage[english]{babel} % Specify a different language here - english by default

\usepackage{marvosym, epigraph, subfig, listings, microtype}

\usepackage[sortcites=false,style=authoryear-comp,bibencoding=utf8, natbib=true, firstinits=true, maxcitenames=2, maxbibnames = 99, uniquename=false, backend=bibtex, useprefix=true, backref=false,doi=false,isbn=false,url=false,dashed=true]{biblatex}
\setlength\bibhang{20pt}
\bibliography{references.bib}
\AtEveryBibitem{%
	\clearfield{day}%
	\clearfield{month}%
	\clearfield{endday}%
	\clearfield{endmonth}%
}

%----------------------------------------------------------------------------------------
%	COLUMNS
%----------------------------------------------------------------------------------------

\setlength{\columnsep}{0.55cm} % Distance between the two columns of text
\setlength{\fboxrule}{0.75pt} % Width of the border around the abstract

%----------------------------------------------------------------------------------------
%	COLORS
%----------------------------------------------------------------------------------------

\definecolor{color1}{RGB}{0,0,90} % Color of the article title and sections
\definecolor{color2}{RGB}{0,20,20} % Color of the boxes behind the abstract and headings

%----------------------------------------------------------------------------------------
%	HYPERLINKS
%----------------------------------------------------------------------------------------

\usepackage{hyperref} % Required for hyperlinks
\hypersetup{hidelinks,colorlinks,breaklinks=true,urlcolor=color2,citecolor=color1,linkcolor=color1,bookmarksopen=false,pdftitle={Housing market and migration revisited: a Bayesian multilevel gravity model for Dutch municipalities},pdfauthor={Thomas de Graaff}}

%----------------------------------------------------------------------------------------
%	ARTICLE INFORMATION
%----------------------------------------------------------------------------------------

\JournalInfo{Conference paper} % Journal information 
\Archive{Prepared for ERSA 2019} % Additional notes (e.g. copyright, DOI, review/research article)

\PaperTitle{Housing market and migration revisited: a Bayesian multilevel gravity model for Dutch municipalities}

\Authors{Thomas de Graaff\textsuperscript{1}*} % Authors
\affiliation{\textsuperscript{1}\textit{Department of Spatial Economics, Vrije Universiteit Amsterdam, Amsterdam, The Netherlands}} % Author affiliation
\affiliation{*\textbf{Corresponding author}: \Letter{} t.de.graaff@vu.n; \Mundus{} \href{thomasdegraaff.nl}{thomasdegraaff.nl}} % Corresponding author

\Keywords{Gravity model --- housing market --- migration --- multilevel model --- partial pooling --- prediction}
\newcommand{\keywordname}{Keywords} 

%%----------------------------------------------------------------------------------------
%%	ABSTRACT
%%----------------------------------------------------------------------------------------

\Abstract{This paper revisits the impact of the housing market
	structure on intercity migration, by applying a Bayesian multilevel gravity model on intercity migration flows. Where most of the existing literatures focuses on using fixed effects for
	cities of origin and destination, I adopt a multilevel mixed effects
	approach allowing me to simultaneously model the impact of migration flow
	characteristics and origin and destination specific effects. This approach has two main advantages. First, it  allows for simultaneous estimation of city specific effects and the effects of city specific home-ownership and social renting rates on migration flows, where the impact is not
	necessarily symmetrical for cities of origin and
	destination. Second, it allows for prediction of migration flows
	between cities both in- and out-of-sample. Preliminary results show
	that home-ownership rates decrease migration flows significantly with an
	elasticity below $-1$. Municipal social renting rate has a negative
	impact as well, but its elasticity is close to zero.}

%----------------------------------------------------------------------------------------
\hypersetup{draft} 
\begin{document}
	
	\flushbottom % Makes all text pages the same height
	\maketitle % Print the title and abstract box
	%\tableofcontents % Print the contents section
	\thispagestyle{empty} % Removes page numbering from the first page
	
	%----------------------------------------------------------------------------------------
	
	\section{Introduction} % The \section*{} command stops section numbering

        In the 1990s, Andrew Oswald wrote two influential working papers
        \citep{oswald1996conjecture, oswald1999housing} postulating
        that home-ownership rates would have a negative impact on labor
        market performance, as the high costs of moving residence
        associated with home-ownership would impede regional
        mobility. These two working papers evoked a large empirical
        literature \citep[see, e.g., ][]{munch2006homeowners,
          munch2008home, de2013european} looking at the impact of
        individual and aggregate home-ownership on labour market
        performance, where seemingly paradoxically at the aggregate
        level home-ownership is indeed harmful for labour market
        behaviour where at the individual level it is correlated with
        positive labour market performance.
        
        This difference between individual and aggregate level is explained by
        sorting. Home-owners are indeed less mobile than private renters
        because of higher fixed and sunk moving costs which has a
        negative \emph{aggregate} effect on labour market
        performance. However, home-owners are different from renters
        as they do \emph{individually} better on the labour market (due to
        individual unobserved heterogeneity). So home-owners in countries with high home-ownership rates perform worse on the labour market vis-\`a-vis home-owners in countries with low home-ownership rates; but they still perform better than private renters. For social renters, the effect is different from private renters. On the individual level they are less mobile than renters at the free market as well, but their labor market performance is also worse than private renters \citep{hughes1981council, de2009homeownership}.
        
        This paper revisits the role of housing market structure as impediment for intercity migration and specifically focuses on the role of home-ownership and social renting rates. To this end, I adopt a Bayesian multi-level
        modelling approach which is not frequently encountered in the
        gravity literature\footnote{That is, in the economic literature; a notable exception is in a trade
        context \citet{ranjan2007bayesian}. In the geographical literature this approach is more commonly adopted \citep[see within a migration context][]{congdon2010random, congdon2012spatial}}. Traditional gravity modelling
        has the disadvantage that either regional fixed effects of
        origins and destinations can be incorporated or the regions'
        characteristics when not varying over flows. This paper circumvents this disadvantage by adopting a
        multilevel approach with partial pooling\footnote{There is a
        whole variety of names for these types of models, including
        varying effects, mixed effects and shrinkage models. I use
        the more generic multilevel description as regions and flows
        are by definition measured at a different level (scale).},
        where the latter terms indicates that I do not impose fixed effects to control for
        origin and destination specific effects, but that I ``draw''
        them from a distribution, hence the name partial pooling
        (where complete pooling states no group effects and no pooling
        fixed effects).
               
        This papers adds two main elements to the
        literature. First, it does not only consider home-ownership
        but as well municipal social renting structure, which can be
        argued \citep[see, e.g.,][]{boyle1998migration, hughes1981council} to have a large effect on regional mobility as well as
        social renting rights are usually only valid locally (within
        municipality) and are lost when moving residence between
        municipalities.
        
        Secondly, A partial pooling approach has another advantage, namely the
        regional varying effects are completely probabilistic, making
        it feasible to predict both within and out-of-sample. In other
        words, with the results at hand I can predict migrations flows between
        existing \emph{and} hypothetical regions. The former might be used for looking at counterfactuals; for example, the changes in in-migration for all municipalities, when one municipality changes its housing structure. The latter is useful when one wants to assess new migrations flows between one or even two new municipalities outside the sample.\footnote{See for probabilistic predictions of internation migration \cite{azose2015bayesian}.}
        
        That housing market structure has a sizeable effect on migration
        decisions is empirically well-established, especially at the micro-level,
        where it is widely accepted that home-ownership has a negative
        effect on regional mobility \citep{dietz2003social, dohmen2005housing}. For
        example, \citet{palomares2018understanding} find that
        home-ownership has a very strong immobility effect on
        internal migration in Spain during the period 2001--2011.
        
        In the literature, less attention has been given to interregional migration on the aggregate level with respect to the housing market as a specific barrier.\footnote{See \citet{cushing2004crossing} for a historical overview of common themes within migration research.} For the UK , \citet{congdon2010random} found within a multilevel gravity model that social rented housing had little effect on the attractivity of a region, although it had a small positive effect on preventing people from moving residence. For the Canadian case, \citet{amirault2016drags} looked at the impact of home-ownership on migration flows within a gravity model using a Poisson pseudo maximum likelihood estimator and found an elasticity around $-1$. 

		The anticipate the results of this paper, I find strong negative effects of home-ownership rates on inter-municipal migration flows. Further, social renting rates also affect migration flows negatively, but the effect is far less pronounced than for home-ownership and overlaps zero to a large extent. A possible interpretation of this finding is that those who sort into social renting are by definition less mobile than those who sort into home-ownership \citep[this argument is put forward by][as well]{boyle1998migration}. Finally, I show that a 0.1 decrease in the rate of home-ownership in Amsterdam leads to in increase of x in-migration y out-migration and significant changes as well in the neighbouring municipaties. 

        This paper reads as follows. The next section describes the
        data and focuses especially on the distribution of regional
        migration flows and regional labour market structure. Section
        4 describes the modelling approach, where starting from
        traditional gravity model and using the descriptives of the
        migration flows, a Bayesian multilevel gravity model is constructed. Section
        5 gives both the model results and interprets them by providing as well predictions within and
        out-of-sample. The last section concludes.

        \section{Data}

        \begin{figure*}[t!]\centering % Using \begin{figure*} makes the figure take up the entire width of the page
          \includegraphics[width=0.8\linewidth]{../fig/hist_mig.pdf}
          \caption{Histogram of migrant flows. Left panel shows the
            histogram of small migrant flows ($0 \leq N < 20$) and the right
            panel shows the histogram of large migrant flows
            ($N \geq 20$). Note the different scale of the y-axes.}
          \label{fig:hist_mig}
        \end{figure*}


        I use inter-municipal migration flows measured in individuals
        between all of the the 393 Dutch municipalities in 2015. There
        is no information available on within municipality residential
        migration. So, I have 393 regional characteristics (or doubled
        when accounting for both regions of origin and destination)
        and 154,056 aggregate migration flows ($393 \times 393 - 393$).

        Figure \ref{fig:hist_mig} shows the distribution of migrant
        flows within my sample. The left panel deals with migrant
        flows below 20, the right panel with migrant flows of 20 and
        larger. Two main observations can be made.

        First, there is strong but consistent decay in migration flow size in both panels,
        which points to a persistent underlying pattern. However, the
        right `tail' in this distribution is rather thick.\footnote{The
          largest migration flows are between the municipalities of
          Amsterdam and Amstelveen and amount to about 3,500
          migrants.} Thus, there are still observations quite far
        right in the distribution. Indeed, the sample mean is about
        10, while the sample variance is around 40, leading to a
        strong presence of \emph{overdispersion} (unconditional on
        other explanatory variables).  Secondly, two thirds of the
        dataset consists of zero observations. Although they do seem
        to be genuine observations and not caused by another process
        (this will be checked later), they do need to be taken
        specifically into account.

        Zeven explanatory variables are added to the model. First, to account for
        spatial distance decay between origin $i$ and destination $j$,
        distance between all municipalities are calculated as
        Eucledian distance between centroids
        ($\text{dist}_{ij}$). Secondly, as municipality mass we use
        population size for both city of origin and city of
        destination (so $\text{pop}_i$ and $\text{pop}_j$). Finally,
        for housing market structure we use variables indicating
        percentage of homeownership ($\text{home}_i$ and
        $\text{home}_j$ and percentage of social renting
        ($\text{soc}_i$ and $\text{soc}_j$), again in both cities of
        origin and destination. Social renting in the Netherlands
        includes all kinds of rent controlled housing but typically
        involves local housing corporations offering housing to lower
        income households, where eligibility is based on (local)
        waiting lists. Both social renting and homeownership are
        assumed to impede regional mobility in the Netherlands as argued in
        \citet{de2009homeownership}.

        \begin{figure*}[ht]\centering % Using \begin{figure*} makes the figure take up the entire width of the page
          \includegraphics[width=0.8\linewidth]{../fig/hist_housing.pdf}
          \caption{Histogram of social housing (left) and
            homeownership (right) percentages in Dutch municipalities
            2015}
            \label{fig:housing_mig}
        \end{figure*}

        Figure \ref{fig:housing_mig} shows the distribution of social
        renting and homeownership across Dutch municipalities in 2015.
        Clearly, both home-ownership and social housing are prevalent across Dutch municipalities, with an average of 25\% of social housing and
        around 60\% of homeownership. Moreover, it is worthwhile to
        note that social renting is especially prevalent in the
        larger cities with a correlation of 0.4 between city size and
        social renting (e.g., Amsterdam has about a 40\% social renting
        rate). Also, some smaller dutch municipalities do not exhibit
        any social renting. Homeownership and city size correlate
        negatively ($-0.51$). Finally, there is a large negative
        correlation between social renting and homeownership ($-0.84$) across municipalities.
        
        \section{Modeling framework}

        \subsection{The traditional gravity model}

        A workhorse model to study aggregate migration flows has throughout time been the gravity model (see \citet{anderson2011gravity} for a generic survey of the use of gravity models and \citet{poot2016gravity} for an overview of migration applications). I therefore start by adopting the basic gravity model specification pioneered by
        \citet{tinbergen1962shaping}, so:
        \begin{equation}
          \text{migrants}_{ij} = \text{pop}_i^{\beta_1}\text{pop}_j^{\beta_2}\text{dist}_{ij}^\gamma
          \label{eq:grav}
        \end{equation}
        Note, that in model (\ref{eq:grav}) the variable
        $\text{dist}_{ij}$ may represent in general all sorts of frictions, not
        only physical distance.
        
        Importantly, \citet{anderson2003gravity} argued that origin
        and destination specific variables should be incorporated to
        take into account multilateral resistance terms. Most often,
        this is done by log-linearising model
        (\ref{eq:grav})\footnote{In our case, note that zeros are
          present in our social renting variable. We therefore add a
          small number to this variables (0.0001). Doing this only on
          the \emph{right-hand side} does only marginally affect our results.} and
        incorporating fixed effects for origins and destinations, as
        follows:
        \begin{equation}
          \log(\text{migrants}_{ij}) = o_i + d_j +  \gamma\log(\text{dist}_{ij})
          \label{eq:gravfixed}
        \end{equation} 
		Note that now all origin and destination specific variables are absorbed by the fixed effects $o_i$ and $d_j$ and that only variables affecting the frictions  $(\text{dist}_{ij})$ can be incorporated.\footnote{If there is another variable dimension---say, repeated observations over time---then this might be circumvented. However, this requires enough variation in the data as time-invariant variables can still not be taken into account.}\footnote{An often applied strategy is to use differences between origin and destination specific variables. Take for example $\Delta h_{ij}$ as the difference in home-ownership rates between $i$ and $j$. A disadvantage of this approach is that the difference between 10\% and 20\% home-ownership rates and the difference between 80\% and 90\% home-ownership rates would be valued as the same.}   
        \begin{figure}[ht]\centering 
        	\includegraphics[width=\linewidth]{../fig/gravity_network.pdf}
        	\caption{Decomposition of variables impacting migration flows from $i$ to $j$ $\left(\{i,j\} \in \{1,\ldots, R\}\right)$}
        	\label{fig:gravity_network}
        \end{figure} 
        Figure \ref{fig:gravity_network} denotes the problem schematically. Typically, one wants to model migration flows between $i$ and $j$, whilst taken into account both the regional specific effects ($o_i$ and $d_j$) and the regional variables ($\mathbf{X}_i$ and $\mathbf{X}_j$) one is interested  in, such as housing market, population structures and cultural variables. 
                  
        Moreover, equation
        (\ref{eq:gravfixed} is typically estimated with regression
        type of models, which is often very cumbersome given the large
        amount of zeros migrants flows. `Quick and dirty' remedies as adding a small amount to the flow variable or removing all zeros have been proven to seriously bias the parameters \citep{linders2006estimation, burger2009specification}.
        	

        Therefore, I next allow for a different strategy, where I
        would like to tackle simultaneously the two disadvantes of above:
        incorporating both city varying effects and city specific variables and
        modelling the distribution of migrants flows as they are
        displayed in Figure \ref{fig:hist_mig}---even when being zero.

        \subsection{A Bayesian multilevel gravity model}

        Firstly, as regional migrants flows are discrete and
        relatively rare give the size of the population, the most
        appropriate way to go forward is to model number of migrants
        with a Poisson type of model. However, given that the sampling
        variance is four times the sampling mean of the migration
        flows (although not conditional on the covariates), we likely
        need to correct for overdispersion of heteroskedasticity
        \citep[][states that heteroskedasticity (rather than the
        presence of too many zeros) is responsible for the main
        differences.]{silva2006log}. An often used distribution to
        account for overdispersion is the gamma-poisson model (also
        known as the negative binomial model). So, we use that for our
        outcome variable.

        To account for the multiplicative nature of the theoretical
        model as in (\ref{eq:grav}), I adopt a log-link for the
        expectation variable in the Poisson model.

        Finally, to adopt both region effects and variables I adopt a
        multilevel model with partial pooling. This entails that our
        regional varying effects (the formerly fixed effecs) are now
        drawn from a, in this case Normal, distribution, where the
        parameters of this distribution are estimated as well (in the
        literature they are known as well as
        hyper-parameters). Intuitively, this entails that regions are
        partially pooled indicating that information between regions
        is shared. This is very attractive, as fixed effects assume no
        pooling. In that case, the model only learns from the information contained
        in that specific region whereas with partial pooling it is ensured that
        outliers (very high or low effects) are effectively
        \emph{shrunk} towards the mean. Indeed, this is a further
        extension of that best feature of linear regression:
        regression towards the mean.

        The total model looks now as follows:
        
        \begin{subequations}
          \begin{align} \text{Migrants}_{ij} \sim & \text{ GammaPoisson}(\lambda_{ij}, \tau) \label{outcome}\\
            \log(\lambda_{ij}) =
            & \alpha + o_{\text{mun}[i]} + d_{\text{mun}[j]} + \notag
            \\ & \beta_1 \log(\text{pop}_i) +
            \beta_2\log(\text{pop}_j) + \notag \\ & \beta_3
            \log(\text{home}_i) + \beta_4 \log(\text{home}_j) + \notag\\
            & \beta_5 \log(\text{soc}_i) + \beta_6 \log(\text{soc}_j)
            + \notag \\ & \beta_7 \log(\text{dist}_{ij}) \label{linear} \\
            o_{\text{mun}} \sim& \text{ Normal}(\alpha_o, \sigma_o) \label{muno} \\
            d_{\text{mun}} \sim& \text{ Normal}(\alpha_d, \sigma_d) \label{mund} \\
            \beta_1,\ldots, \beta_7 \sim& \text{
                                          Normal}(0,2)\\ \alpha_o, \alpha_d \sim& \text{ Normal}(0,2)\\
            \sigma_o, \sigma_d \sim& \text{ HalfCauchy}(0,1) \\ \tau
            \sim& \text{ Gamma}(0.01, 0.01)
          \end{align}
          \label{model}
        \end{subequations}

        The first line ({\ref{outcome}) models the outcome variable,
          being the number of migrants, using a Poisson distribution
          (with parameter $\lambda_{ij}$) allowing for overdispersion
          by using an additional parameter $\tau$. The linear part of
          the model is given by (\ref{linear}) and states that the
          poisson outcome variable is on a log-scale and that most
          variables are on a log-scale as well, allowing for direct
          comparison of the parameters being elasticities. Equations
          (\ref{muno}) and {(\ref{mund}) constitute the multilevel
            part, where parameters $\sigma_o$ and $\sigma_d$ measure
            the amount of pooling. If they tend to zero, then the
            data exhibits complete pooling. If they become very large
            (go to infinity) there is no pooling (thus fixed
            effects). All the other parameters are priors (chosen such
            that they are rather conservative but given the amount of
            data they are of little influence).
          
        \section{Results}

        \subsection{Parameter estimates}
        
        Model (\ref{model}) is estimated by using the \emph{No U-Turn
          Sampler} (NUTS) from the Stan application.\footnote{See \href{https://mc-stan.org/}{https://mc-stan.org/}. As interface to Stan
          \citep[see for an overview article of
          Stan][]{carpenter2017stan} I used the R-package \citep{brms} \texttt{brms}.} NUTS is a relatively recent developed Hamiltonian Monte Carlo (a specific form of Markov
        Chain Monte Carlo simulation) method, able to draw samples
        efficiently from large multilevel models
        \citep{hoffman2014no}. Parameter estimates and probability
        intervals of the main parameters (so not the region specific
        effects: there are 786 of them) are given in Table
        \ref{tab:coef}. Perhaps more insightful, they are as well graphically
        depicted in Figure \ref{fig:forestplot}.

% latex table generated in R 3.4.4 by xtable 1.8-3 package
% Fri Feb 22 15:07:02 2019
\begin{table}[ht]
  \centering
  \caption{Parameter estimates with 95\% probability intervals (group specific origin and destination estimates are not presented)}
  \label{tab:coef}
  \begin{tabular}{lrrrr}
    \toprule
    Parameter & mean & sd & 2.5\% & 97.5\% \\ 
    \midrule
    Intercept      & $-0.74$ & 0.04 & $-0.82$ & $-0.66$ \\ 
    log(pop$_i$)   & 0.89 & 0.03 & 0.83 & 0.96 \\ 
    log(pop$_j$)   & 0.88 & 0.04 & 0.79 & 0.97 \\ 
    log(home$_i$)  & $-1.48$ & 0.19 & $-1.86$ & $-1.10$ \\ 
    log(home$_j$)  & $-1.27$ & 0.25 & $-1.75$ & $-0.78$ \\ 
    log(soc$_i$)   & $-0.04$ & 0.04 & $-0.11$ & 0.03 \\
    log(soc$_j$)   & $-0.06$ & 0.03 & $-0.12$ & $-0.01$ \\ 
    log(dist$_{ij})$ & $-1.96$ & 0.01 & $-1.97$ & $-1.95$ \\ 
    $\sigma_o$    & 0.45 & 0.02 & 0.42 & 0.49 \\ 
    $\sigma_j$    & 0.61 & 0.02 & 0.57 & 0.66 \\ 
    $\tau$        & 1.22 & 0.01 & 1.20 & 1.24 \\ 
    \bottomrule
  \end{tabular}
\end{table}

\begin{figure}
  \includegraphics[width = \columnwidth]{../fig/forestplot.pdf}
  \caption{Forest plot of parameter means and 95\% probability
    intervals (group specific origin and destination estimates are not
    presented)}
  \label{fig:forestplot}
\end{figure}

Obviously, housing structure indeed impedes regional mobility, but that it is primarily 
home-ownership rates and not social renting rates that have a negative
effect. The home-ownership elasticities are slightly larger in absolute
size than what \citet{amirault2016drags} reported. Furthermore, if anything, 
estimations for parameters $\sigma_o$ and $\sigma_d$ point to more
pooling than less, so fixed effects in this case could lead to
substantial overfitting. 

The other variables are conform previous literature. The elasticities of population for both regions of origin and destination are around 0.9 and the distance-decay parameter is $-1.96$. The latter is significantly larger in absolute value that typical parameters reported by regression type of models, but is more in less line with parameters estimated by negative binomial or poisson type of models. 

For each municipalities the models estimates as well an origin specific effect ($o_i$) and a destination specific effect ($d_j$). These are depicted in maps \ref{fig:out} and \ref{fig:in} and can be interpreted as having relatively less or more migration \emph{relative} to population size and housing structure.

\begin{figure}[h!]
		\centering
		\includegraphics[width = \columnwidth]{../fig/p_coef_out.pdf}
		\caption{Region specific origin effects}\label{fig:out}
\end{figure}

Figure \ref{fig:out} first shows the region specific origin effects. Clearly, there is a relatively larger push factor in the peripheral Dutch municipalities. These are actually regions that loose population. The areas that loose relatively few people are to be found in the Dutch core region (the ``Randstad'') including the largest Dutch cities: Amsterdam, Rotterdam, The Hague and Utrecht.  

\begin{figure}[h!]
	\centering
	\includegraphics[width = \columnwidth]{../fig/p_coef_in.pdf}
	\caption{Region specific destination effects}\label{fig:in}
\end{figure}

Figure \ref{fig:in} shows the region specific destination effects. Interestingly, these seem to mirror to a certain extent the region specific origin effects. Again, relatively most people seem to move to the peripheral municipalities---albeit to a lesser extent than the push factor. That means that controlling for housing and population, relatively few people move to the largest cities in the Netherlands. This is due to the scarcity of available housing in that period (partly due to the housing structure) and the fact that this model does not reflect price differences. The ``Randstad area'' in the Netherlands is very popular, but because of housing shortage this is mainly reflected in higher prices and an increasingly tighter housing market. Housing is still available in the least attractive regions of the Netherlands and this is where most dynamics in terms of moving residence take place. This indicates as well that the region specific destination effects should not be interpreted as attractivity effects as, e.g., in \citet{congdon2010random}.

Figures \ref{fig:out} and \ref{fig:in} indicates as well that the region specific origin and destination effects are most likely highly correlated. Moreover, given the facts that nearby municipalities often show similar values, ideally both types of effects should be modelled as being spatially autocorrelated. 

\subsection{Model predictions}

\section{In conclusion}

\section*{Acknowledgments} % The \section*{} command stops section numbering

I would like to thank Wim Bernasco for valuable comments on a first
draft of this paper. Paper, data and code can be retrieved from the
project's GitHub page:

\href{https://github.com/Thdegraaff/migration_gravity}{https://github.com/Thdegraaff/migration\_gravity}.
	%----------------------------------------------------------------------------------------
	%	REFERENCE LIST
	%----------------------------------------------------------------------------------------
	
	\addcontentsline{toc}{section}{references} % Adds this section to the table of contents
	\printbibliography
	
	%----------------------------------------------------------------------------------------
	
\end{document}
%%% Local Variables:
%%% mode: latex
%%% TeX-master: t
%%% End:
