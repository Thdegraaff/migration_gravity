%%%%%%%%%%%%%%%%%%%%%%%%%%%%%%%%%%%%%%%%%
% Stylish Article
% LaTeX Template
% Version 2.1 (1/10/15)
%
% This template has been downloaded from:
% http://www.LaTeXTemplates.com
%
% Original author:
% Mathias Legrand (legrand.mathias@gmail.com) 
% With extensive modifications by:
% Vel (vel@latextemplates.com)
%
% License:
% CC BY-NC-SA 3.0 (http://creativecommons.org/licenses/by-nc-sa/3.0/)
%1
%%%%%%%%%%%%%%%%%%%%%%%%%%%%%%%%%%%%%%%%%

%--------------------------------------------------------------# Path to your oh-my-zsh installation.--------------------------
%	PACKAGES AND OTHER DOCUMENT CONFIGURATIONS
%----------------------------------------------------------------------------------------

\documentclass[fleqn,10pt]{SelfArx} % Document font size and equations flushed left

\usepackage[english]{babel} % Specify a different language here - english by default

\usepackage{marvosym, epigraph, subfig, listings}

\usepackage[sortcites=false,style=authoryear-comp,bibencoding=utf8, natbib=true, firstinits=true, maxcitenames=2, maxbibnames = 99, uniquename=false, backend=bibtex, useprefix=true, backref=false,doi=false,isbn=false,url=false,dashed=true]{biblatex}
\setlength\bibhang{20pt}
\bibliography{references.bib}
\AtEveryBibitem{%
	\clearfield{day}%
	\clearfield{month}%
	\clearfield{endday}%
	\clearfield{endmonth}%
}

%----------------------------------------------------------------------------------------
%	COLUMNS
%----------------------------------------------------------------------------------------

\setlength{\columnsep}{0.55cm} % Distance between the two columns of text
\setlength{\fboxrule}{0.75pt} % Width of the border around the abstract

%----------------------------------------------------------------------------------------
%	COLORS
%----------------------------------------------------------------------------------------

\definecolor{color1}{RGB}{0,0,90} % Color of the article title and sections
\definecolor{color2}{RGB}{0,20,20} % Color of the boxes behind the abstract and headings

%----------------------------------------------------------------------------------------
%	HYPERLINKS
%----------------------------------------------------------------------------------------

\usepackage{hyperref} % Required for hyperlinks
\hypersetup{hidelinks,colorlinks,breaklinks=true,urlcolor=color2,citecolor=color1,linkcolor=color1,bookmarksopen=false,pdftitle={What drives which region?},pdfauthor={Thomas de Graaff}}

%----------------------------------------------------------------------------------------
%	ARTICLE INFORMATION
%----------------------------------------------------------------------------------------

\JournalInfo{Conference paper} % Journal information 
\Archive{Prepared for ERSA 2019} % Additional notes (e.g. copyright, DOI, review/research article)

\PaperTitle{Housing market and migration revisited: a multilevel gravity model for Dutch municipalities}

\Authors{Thomas de Graaff\textsuperscript{1}*} % Authors
\affiliation{\textsuperscript{1}\textit{Department of Spatial Economics, Vrije Universiteit Amsterdam, Amsterdam, The Netherlands}} % Author affiliation
\affiliation{*\textbf{Corresponding author}: \Letter{} t.de.graaff@vu.n; \Mundus{} \href{thomasdegraaff.nl}{thomasdegraaff.nl}} % Corresponding author

\Keywords{Gravity model --- housing market --- migration --- multilevel model --- partial pooling --- prediction}
\newcommand{\keywordname}{Keywords} 

%%----------------------------------------------------------------------------------------
%%	ABSTRACT
%%----------------------------------------------------------------------------------------

\Abstract{This paper revisits the impact of the housing market
  structure on interregional migration, but adopts an alternative
  modeling approach to regional migration flows. The starting
  point is a gravity model, but instead of using fixed effects for
  cities of origin and destination, I use a multilevel mixed effects
  approach allowing me to simultaneously model migration flow
  characteristics and the cities of origin and destination
  varying effects. This approach has two main advantages. First, it
  allows for simultaneous estimation of the impact of city
  characteristics on migration flows, where the impact is not
  necessarily symmetrical for cities of origin and
  destination. Second, it allows for prediction of migration flows
  between cities both in and out of sample. Preliminary results show
  that homeownership decrease migration flows significantly with an
  elasticity below $-1$. Municipal social renting rate has a negative
  impact as well, but its elasticity is close to zero.}

%----------------------------------------------------------------------------------------
\hypersetup{draft} 
\begin{document}
	
	\flushbottom % Makes all text pages the same height
	\maketitle % Print the title and abstract box
	%\tableofcontents % Print the contents section
	\thispagestyle{empty} % Removes page numbering from the first page
	
	%----------------------------------------------------------------------------------------
	
	\section{Introduction} % The \section*{} command stops section numbering

        In the 1990s, Andrew Oswald wrote two famous working papers
        \citep{oswald1996conjecture, oswald1999housing} postulating
        that home-ownership rates would have a negative impact on labor
        market behaviour, as the high costs of moving residence
        associated with home-ownership would impede regional
        mobility. These two working papers evoked a large empirical
        literature \citep[see, e.g., ][]{munch2006homeowners,
          munch2008home, de2013european} looking at the impact of
        individual and aggregate home-ownership on labour market
        performance, where seemingly paradoxically at the aggregate
        level home-ownership is indeed harmful for labour market
        behaviour where at the individual level it is correlated with
        positive labour market performance.
        
        Theoretically, this difference is explained by sorting. Home-owners are indeed less mobile than renters because of higher fixed and sunk moving costs which has a negative aggregate effect on 
        labour market performance. However, home-owners are different from renters as they do individually
        better on the labour market (due to unobservables). 

        That housing market structure has an effect on migration
        decisions is empirically well-established, especially at the micro-level,
        where it is widely accepted that home-ownership has a negative
        effect on regional mobility \citep{dietz2003social}. For
        example, \citet{palomares2018understanding} find that
        home-ownership has a very strong immobility effect on
        internal migration in Spain during the period 2001--2011.

        On an aggregate level, \citet{amirault2016drags}, amongst others,
        looked at the impact of home-ownership on migration flows
        within a gravity model using a Poisson pseudo maximum
        likelihood estimator and found an elasticity around
        $-1$. 
        
        This papers revisits the relation between housing markets and regional migration and adds two main elements to the literature. First, it does not only consider home-ownership but as well municipal social renting structure, which can be argued to have a huge effect on regional mobility as well as social renting rights are usually only valid locally (within municipality) and are lost when moving residence between municipalities.
        
        Secondly, we adopt an alternative Bayesian multi-level modelling approach which is not frequently encountered in the gravity literature \citep[a notable exception is][in a trade context]{ranjan2007bayesian}. Traditional gravity modelling has the
        disadvantage that either regional fixed effects of origins and
        destinations can be incorporated or the regions'
        characteristics when not varying over flows. Moreover,
        theoretically, regional effects should be incorporated leaving
        no room in the traditional approach to incorporate regional characteristics

        This paper circumvents this disadvantage by adopting a
        multilevel approach with partial pooling\footnote{There is a
          whole variety of names for these types of models, including
          varying effects, mixed effects and shrinkage models. I use
          the more generic multilevel description as regions and flows
          are by definition measured at a different level (scale).},
        where the latter terms indicates that I adopt regions of
        origin and destination specific effects, but that I ``draw''
        them from a distribution, hence the name partial pooling
        (where complete pooling states no group effects and no pooling
        fixed effects).

        A partial pooling approach has another advantage, namely the
        regional varying effects are completely probabilistic, making
        it feasible to predict both within and out-of-sample. In other
        words, with the results at hand I can predict flows between
        existing \emph{and} hypothetical regions.

        This paper reads as follows. The next section describes the
        data and focuses especially on the distribution of regional
        migration flows and regional labour market structure. Section
        3 describes the modelling approach, where starting from
        traditional gravity model and using the descriptives of
        migration flows I argue for a specific type of model. Section
        4. gives both the model results and their analysis. By the
        latter I mean that this sections deals as well with
        interpretation by giving prediction both within and
        out-of-sample. The last section concludes.

        \section{Data}

        \begin{figure*}[ht!]\centering % Using \begin{figure*} makes the figure take up the entire width of the page
          \includegraphics[width=0.8\linewidth]{../fig/hist_mig.pdf}
          \caption{Histogram of migrant flows. Left panel shows the
            histogram of small migrant flows ($N<0$) and the right
            panel shows the histogram of large migrant flows
            ($N \geq 20$). Note the different scale of the y-axis.}
          \label{fig:hist_mig}
        \end{figure*}


        I use inter-municipal migration flows measured in individuals
        between all of the the 393 Dutch municipalities in 2015. There
        is no information available on within municipality residential
        migration. So, I have 393 regional characteristics (or doubled
        when accounting for both regions of origin and destination)
        and 154,056 flows ($393 \times 393 - 393$).

        Figure \ref{fig:hist_mig} shows the distribution of migrant
        flows within my sample. The left panel deals with migrant
        flows below 20, the right panel with migrant of 20 and
        larger. Two main observations can be made.

        First, there is strong but consistent decay in both panels,
        which points to a persistent underlying pattern. However, the
        `tail' in this distribution is rather thick.\footnote{The
          largest migration flows are between the municipalities of
          Amsterdam and Amstelveen and amount to about 3,500
          migrants.} Thus, there are still observations quite far
        right in the distribution. Indeed, the sample mean is about
        10, while the sample variance is around 40, leading to a
        strong presence of \emph{overdispersion} (unconditional on
        other explanatory variables).  Secondly, two thirds of the
        dataset consists of zero observations. Although they do seem
        to be genuine observations and not caused by another process
        (we will check for this later), they do need to be taken
        specifically into account.

        I include 7 other variables in my model. First, to account for
        spatial distance decay between origin $i$ and destination $j$,
        distance between all municipalities are calculated as
        Eucledian distance between centroids
        ($\text{dist}_{ij}$). Secondly, as municipality mass we use
        population size for both city of origin and city of destination (so $\text{pop}_i$ and
        $\text{pop}_j$). Finally, for housing market structure we use
        variables indicating percentage of homeownership
        ($\text{home}_i$ and $\text{home}_j$ and percentage of social
        renting ($\text{soc}_i$ and $\text{soc}_j$), again in both cities of origin and destination. Social renting in the Netherlands includes all kinds of rent controlled housing but typically
        involves local housing corporations offering housing to lower
        income households, where eligibility is based on (local) waiting
        lists. Both social renting and homeownership are assumed to
        impede regional mobility as argued in \citep{de2009homeownership}.

        \begin{figure*}[ht]\centering % Using \begin{figure*} makes the figure take up the entire width of the page
          \includegraphics[width=0.8\linewidth]{../fig/hist_housing.pdf}
          \caption{Histogram of social housing (left) and
            homeownership (right) percentages in Dutch municipalities
            2015}
            \label{fig:housing_mig}
        \end{figure*}

        Figure \ref{fig:housing_mig} shows the distribution of social
        renting and homeownership across Dutch municipalities in 2015.
        Clearly, both types of housing structures are important for
        the Netherlands, with an average of 25\% of social housing and
        around 60\% of homeownership. Moreover, it is worthwhile to
        note that social renting is especially prevalent in the the
        larger cities with a correlation of 0.4 between city size and
        social renting (e.g., Amsterdam has about 40\% social renting
        rate). Also, some smaller dutch municipalities do not exhibit
        any social renting. Homeownership and city size correlate
        negatively ($-0.51$). Finally, there is a large negative
        correlation between social renting and homeownership ($-0.84$) across municipalities.
        
        \section{Modeling framework}

        \subsection{The traditional gravity model}

        To start with, I adopt the basic gravity model specification pioneered by
        \citet{tinbergen1962shaping}, so:
        \begin{equation}
          \text{migrants}_{ij} = \text{pop}_i^{\beta_1}\text{pop}_j^{\beta_2}\text{dist}_{ij}^\gamma
          \label{eq:grav}
        \end{equation}
        Note, that in model (\ref{eq:grav}) the variable
        $\text{dist}_{ij}$ may represent all sorts of frictions, not
        only physical distance. Thus, in my case we incorporate
        variables for homeownership and social renting to account for
        frictions on the housing market that may impede regional
        mobility.

        Importantly, \citet{anderson2003gravity} argued that origin
        and destination specific variables should be incorporated to
        take into account multilateral resistance terms. Most often,
        this is done by log-linearising model
        (\ref{eq:grav})\footnote{In our case, note that zeros are
          present in our social renting variable. We therefore add a
          small number to this variables (0.0001). Doing this only on
          the \emph{right-hand side} does not affects our results} and
        incorporating fixed effects for origins and destinations, as
        follows:
        \begin{equation}
          \log(\text{migrants}_{ij}) = o_i + d_j +  \gamma\log(\text{dist}_{ij})
          \label{eq:gravfixed}
        \end{equation} 
        Unfortunately, this approach does not allow for municipality
        specific variables; so, population and housing market
        variables drop out of this model. But those are exactly the
        variables I am  interested in! Moreover, equation
        (\ref{eq:gravfixed} is typically estimated with regression
        type of models, which is often very cumbersome given the large
        amount of zeros migrants flows.

        Therefore, I next allow for a different strategy, where I
        would like to tackle simultaneously the two disadvantes of above:
        incorporating both city varying effects and city specific variables and
        modelling the distribution of migrants flows as they are
        displayed in Figure \ref{fig:hist_mig}---even when being zero.

        \subsection{A multilevel gravity model}

        Firstly, as regional migrants flows are discrete and
        relatively rare give the size of the population, the most
        appropriate way to go forward is to model number of migrants
        with a Poisson type of model. However, given that the sampling
        variance is four times the sampling mean of the migration
        flows (although not conditional on the covariates), we likely
        need to correct for overdispersion of heteroskedasticity
        \citep[][states that heteroskedasticity (rather than the
        presence of too many zeros) is responsible for the main
        differences.]{silva2006log}. An often used distribution to
        account for overdispersion is the gamma-poisson model (also
        known as the negative binomial model). So, we use that for our
        outcome variable.

        To account for the multiplicative nature of the theoretical
        model as in (\ref{eq:grav}), I adopt a log-link for the
        expectation variable in the Poisson model.

        Finally, to adopt both region effects and variables I adopt a
        multilevel model with partial pooling. This entails that our
        regional varying effects (the formerly fixed effecs) are now
        drawn from a, in this case Normal, distribution, where the
        parameters of this distribution are estimated as well (in the
        literature they are known as well as
        hyper-parameters). Intuitively, this entails that regions are
        partially pooled indicating that information between regions
        is shared. This is very attractive, as fixed effects assume no
        pooling. In that case, the model only learns from the information contained
        in that specific region whereas with partial pooling it is ensured that
        outliers (very high or low effects) are effectively
        \emph{shrunk} towards the mean. Indeed, this is a further
        extension of that best feature of linear regression:
        regression towards the mean.

        The total model looks now as follows:
        
        \begin{subequations}
          \begin{align} \text{Migrants}_{ij} \sim & \text{GammaPoisson}(\lambda_{ij}, \tau) \label{outcome}\\
            \log(\lambda_{ij}) =
            & \alpha + o_{\text{mun}[i]} + d_{\text{mun}[j]} + \notag
            \\ & \beta_1 \log(\text{pop}_i) +
            \beta_2\log(\text{pop}_j) + \notag \\ & \beta_3
            \log(\text{home}_i) + \beta_4 \log(\text{home}_j) + \notag\\
            & \beta_5 \log(\text{soc}_i) + \beta_6 \log(\text{soc}_j)
            + \notag \\ & \beta_7 \log(\text{dist}_{ij}) \label{linear} \\
            o_{\text{mun}} \sim& \text{ Normal}(\alpha_o, \sigma_o) \label{muno} \\
            d_{\text{mun}} \sim& \text{ Normal}(\alpha_d, \sigma_d) \label{mund} \\
            \beta_1,\ldots, \beta_7 \sim& \text{
                                          Normal}(0,2)\\ \alpha_o, \alpha_d \sim& \text{ Normal}(0,2)\\
            \sigma_o, \sigma_d \sim& \text{ HalfCauchy}(0,1) \\ \tau
            \sim& \text{ Gamma}(0.01, 0.01)
          \end{align}
          \label{model}
        \end{subequations}

        The first line ({\ref{outcome}) models the outcome variable,
          being the number of migrants, using a Poisson distribution
          (with parameter $\lambda_{ij}$) allowing for overdispersion
          by using an additional parameter $\tau$. The linear part of
          the model is given by (\ref{linear}) and states that the
          poisson outcome space is on a log-scale and that most
          parameters are on a log-scale as well, allowing for direct
          comparison of the parameters being elasticities. Equations
          (\ref{muno}) and {(\ref{mund}) constitute the multilevel
            part, where parameters $\sigma_o$ and $\sigma_d$ measure
            the amount of pooling. If they tend to zero, then the
            data exhibits complete pooling. If they become very large
            (go to infinity) there is no pooling (thus fixed
            effects). All the other parameters are priors (chosen such
            that they are rather conservative but given the amount of
            data they are of little influence).
          
        \section{Results}

        \subsection{Parameter estimates}
        
        I estimate model (\ref{model}) by using the \emph{No U-Turn
          Sampler} (NUTS) from the Stan application.\footnote{See \href{https://mc-stan.org/}{https://mc-stan.org/}. As interface to Stan
          \citep[see for an overview article of
          Stan][]{carpenter2017stan} I used the R-package \citep{brms} \texttt{brms}.} NUTS is a relatively recent developed Hamiltonian Monte Carlo (a specific form of Markov
        Chain Monte Carlo simulation) method, able to draw samples
        efficiently from large multilevel models
        \citep{hoffman2014no}. Parameter estimates and probability
        intervals of the main parameters (so not the region specific
        effects: there are 786 of them) are given in Table
        \ref{tab:coef}. Perhaps more insightful, there are graphically
        depicted in Figure \ref{fig:forestplot}.

% latex table generated in R 3.4.4 by xtable 1.8-3 package
% Fri Feb 22 15:07:02 2019
\begin{table}[ht]
  \centering
  \caption{Parameter estimates with 95\% probability intervals (group specific origin and destination estimates are not presented)}
  \label{tab:coef}
  \begin{tabular}{lrrrr}
    \toprule
    Parameter & mean & sd & 2.5\% & 97.5\% \\ 
    \midrule
    b\_Intercept & -0.74 & 0.04 & -0.82 & -0.66 \\ 
    b\_pop\_d & 0.89 & 0.03 & 0.83 & 0.96 \\ 
    b\_pop\_o & 0.88 & 0.04 & 0.79 & 0.97 \\ 
    b\_hom\_d & -1.48 & 0.19 & -1.86 & -1.10 \\ 
    b\_hom\_o & -1.27 & 0.25 & -1.75 & -0.78 \\ 
    b\_soc\_o & -0.04 & 0.04 & -0.11 & 0.03 \\ 
    b\_soc\_d & -0.06 & 0.03 & -0.12 & -0.01 \\ 
    b\_log\_distance & -1.96 & 0.01 & -1.97 & -1.95 \\ 
    sd\_destination\_\_Intercept & 0.45 & 0.02 & 0.42 & 0.49 \\ 
    sd\_origin\_\_Intercept & 0.61 & 0.02 & 0.57 & 0.66 \\ 
    shape & 1.22 & 0.01 & 1.20 & 1.24 \\ 
    \bottomrule
  \end{tabular}
\end{table}

\begin{figure}
  \includegraphics[width = \columnwidth]{../fig/forestplot.pdf}
  \caption{Forest plot of parameter means and 95\% probability
    intervals (group specific origin and destination estimates are not
    presented)}
  \label{fig:forestplot}
\end{figure}

As most important conclusions in this stage I can say that housing
structure indeed impedes regional mobility, but that it is primarily 
home-ownership rates and not social renting rates that have a negative
effect. The home-ownership elasticities are slightly larger in absolute
size than what \citet{amirault2016drags} reported. Furthermore, if anything, 
estimations for parameters $\sigma_o$ and $\sigma_d$ point to more
pooling than less, so fixed effects in this case might lead to
substantial overfitting.

\subsection{Model predictions}

\section{In conclusion}

\section*{Acknowledgments} % The \section*{} command stops section numbering

I would like to thank Wim Bernasco for valuable comments on a first
draft of this paper. Paper, data and code can be retrieved from the
project's GitHub page:

\href{https://github.com/Thdegraaff/migration_gravity}{https://github.com/Thdegraaff/migration\_gravity}.
	%----------------------------------------------------------------------------------------
	%	REFERENCE LIST
	%----------------------------------------------------------------------------------------
	
	\addcontentsline{toc}{section}{references} % Adds this section to the table of contents
	\printbibliography
	
	%----------------------------------------------------------------------------------------
	
\end{document}
%%% Local Variables:
%%% mode: latex
%%% TeX-master: t
%%% End:
