%%%%%%%%%%%%%%%%%%%%%%%%%%%%%%%%%%%%%%%%%
% Stylish Article
% LaTeX Template
% Version 2.1 (1/10/15)
%
% This template has been downloaded from:
% http://www.LaTeXTemplates.com
%
% Original author:
% Mathias Legrand (legrand.mathias@gmail.com) 
% With extensive modifications by:
% Vel (vel@latextemplates.com)
%
% License:
% CC BY-NC-SA 3.0 (http://creativecommons.org/licenses/by-nc-sa/3.0/)
%1
%%%%%%%%%%%%%%%%%%%%%%%%%%%%%%%%%%%%%%%%%

%--------------------------------------------------------------# Path to your oh-my-zsh installation.--------------------------
%	PACKAGES AND OTHER DOCUMENT CONFIGURATIONS
%----------------------------------------------------------------------------------------

\documentclass[fleqn,10pt]{SelfArx} % Document font size and equations flushed left

\usepackage[english]{babel} % Specify a different language here - english by default

\usepackage{marvosym, epigraph, subfig, listings}

\usepackage[sortcites=false,style=authoryear-comp,bibencoding=utf8, natbib=true, firstinits=true, maxcitenames=2, maxbibnames = 99, uniquename=false, backend=bibtex, useprefix=true, backref=false,doi=false,isbn=false,url=false,dashed=true]{biblatex}
\setlength\bibhang{20pt}
\bibliography{references.bib}
\AtEveryBibitem{%
	\clearfield{day}%
	\clearfield{month}%
	\clearfield{endday}%
	\clearfield{endmonth}%
}

%----------------------------------------------------------------------------------------
%	COLUMNS
%----------------------------------------------------------------------------------------

\setlength{\columnsep}{0.55cm} % Distance between the two columns of text
\setlength{\fboxrule}{0.75pt} % Width of the border around the abstract

%----------------------------------------------------------------------------------------
%	COLORS
%----------------------------------------------------------------------------------------

\definecolor{color1}{RGB}{0,0,90} % Color of the article title and sections
\definecolor{color2}{RGB}{0,20,20} % Color of the boxes behind the abstract and headings

%----------------------------------------------------------------------------------------
%	HYPERLINKS
%----------------------------------------------------------------------------------------

\usepackage{hyperref} % Required for hyperlinks
\hypersetup{hidelinks,colorlinks,breaklinks=true,urlcolor=color2,citecolor=color1,linkcolor=color1,bookmarksopen=false,pdftitle={What drives which region?},pdfauthor={Thomas de Graaff}}

%----------------------------------------------------------------------------------------
%	ARTICLE INFORMATION
%----------------------------------------------------------------------------------------

\JournalInfo{Conference paper} % Journal information 
\Archive{Prepared for ERSA 2019} % Additional notes (e.g. copyright, DOI, review/research article)

\PaperTitle{Housing market and migration revisited: a multilevel gravity model for Dutch municipalities}

\Authors{Thomas de Graaff\textsuperscript{1}*} % Authors
\affiliation{\textsuperscript{1}\textit{Department of Spatial Economics, Vrije Universiteit Amsterdam, Amsterdam, The Netherlands}} % Author affiliation
\affiliation{*\textbf{Corresponding author}: \Letter{} t.de.graaff@vu.n; \Mundus{} \href{thomasdegraaff.nl}{thomasdegraaff.nl}} % Corresponding author

\Keywords{Gravity model --- housing market --- migration --- multilevel model --- partial pooling --- prediction}
\newcommand{\keywordname}{Keywords} 

%%----------------------------------------------------------------------------------------
%%	ABSTRACT
%%----------------------------------------------------------------------------------------

\Abstract{This paper revisits the impact of the housing market
  structure on interregional migration, but adopts an alternative
  modeling approach to migration flows between cities. The starting
  point is a gravity model, but instead of using fixed effects for
  cities of origin and destination, I use a multilevel mixed effects
  approach allowing me to simultaneously model migration flow
  characteristics and the cities of origin and destination
  characteristics. This approach has two main advantages. First, it
  allows for simultaneous estimation of the impact of city
  characteristics on migration flows, where the impact is not
  necessarily symmetrical for cities of origin and
  destination. Second, it allows for prediction of migration flows
  between cities both in and out of sample. Preliminary results show
  that homeownership decrease migration flows significantly with an
  elasticity below $-1$. Municipal social renting rate has a negative
  impact as well, but its elasticity is close to zero.}

%----------------------------------------------------------------------------------------
\hypersetup{draft} 
\begin{document}
	
	\flushbottom % Makes all text pages the same height
	\maketitle % Print the title and abstract box
	%\tableofcontents % Print the contents section
	\thispagestyle{empty} % Removes page numbering from the first page
	
	%----------------------------------------------------------------------------------------
	
	\section{Introduction} % The \section*{} command stops section numbering

        In the 1990s, Andrew Oswald wrote two famous working papers
        \citep{oswald1996conjecture, oswald1999housing} postulating
        that homeownership rates would have a negative impact on labor
        market behavior, as the high costs of moving residence
        associated with homeownership would impede regional
        mobility. These two working papers evoked a large empirical
        literature \citep[see, e.g., ][]{munch2006homeowners,
          munch2008home, de2013european} looking at the impact of
        individual and aggregate homeownership on labor market
        performance, where seemingly paradoxically at the aggregate
        level homeownership is indeed harmful for labor market
        behavior where at the individual level it is correlated with
        positive labor market performance.

        That housing market structure has an effect on migration
        decisions is well-established, especially at the micro-level,
        where it is widely accepted that homeownership has a negative
        effect on regional mobility \citep{dietz2003social}. For
        example, \citet{palomares2018understanding} find that
        homeownership has a very strong immobility effect on in
        internal migration in Spain during the period 2001--2011.

        On an aggregate level, \citet{amirault2016drags} already
        looked at the impact of homeownership on migration flows
        within a gravity model using a Poisson pseudo maximum
        likelihood estimator and found an elasticity around
        $-1$. However, traditional gravity modeling has the
        disadvantage that either regional fixed effects of origins and
        destinations can be incorporated or the regions'
        characteristics when not varying over flows. Moreover,
        theoretically, regional effecs should be incorporated leaving
        no room in the traditional approach to incorporate regional characteristics

        This paper circumvents this disadvanted by adopting a
        multilevel approach with partial pooling\footnote{There is a
          whole variety of names for these types of models, including
          varying effects, mixed effects and shrinkage models. I use
          the more generic multilevel description as regions and flows
          are by definition measured at a different level (scale).},
        where the latter terms indicates that I adopt regions of
        origin and destination specific effects, but that I ``draw''
        them from a distribution, hence the name partial pooling
        (where complete pooling states no group effects and no pooling
        fixed effects.

        A partial pooling approach has another advantege, namely the
        regional specific effects are completely probabilistic, making
        it feasible to predict both within and out-of-sample. In other
        words, with the results at hand I can predict flows between
        hypothetical regions.

        This paper reads as follows. The next section describes the
        data and focuses especially on the distribution of regional
        migration flows and regional labour market structure. Section
        3 describes the modeling approach, where starting from
        traditional gravity model and using the descriptives of
        migration flows I argue for a specific type of model. Section
        4. gives both the model results and their analysis. By the
        latter I mean that this sections deals as well with
        interpretation by giving prediction both within and
        out-of-sample. The last section concludes.

        \section{Data}

        I model the migration flows measured in individuals between
        the 393 Dutch municipalities in 2015. There is no information
        available about within municipality residential migration. So,
        I have 393 regional characteristics (or doubled when
        accounting for both regions of origin and destination) and
        154,056 flows ($393 \times 393 - 393$). 

        \begin{figure*}[ht]\centering % Using \begin{figure*} makes the figure take up the entire width of the page
          \includegraphics[width=0.8\linewidth]{../fig/hist_mig.pdf}
          \caption{Histogram of migrant flows. Left panel shows the
            histogram of small migrant flows ($N<0$) and the right
            panel shows the histogram of large migrant flows
            ($N \geq 20$). Note the different scale of the y-axis.}
          \label{fig:hist_mig}
        \end{figure*}

        Figure \ref{fig:hist_mig} shows the distribution of migrant
        flows within my sample. The left panel deals with migrant
        flows below 20, the right panel with migrant of 20 and
        larger. Two main observations can be made.

        First, there is strong but consistent decay in both panels,
        which points to a strong underlying pattern. However, the
        `tail' in this distribution is rather thick.\footnote{The
          largest migration flows are between the municipalities of
          Amsterdam and Amstelveen and amount to about 3,500
          migrants.} There are still observations quite far right in
        the distributution. Indeed, sample mean is about 10, while the
        sample variance is aournd 40, leading to a strong presence of
        \emph{overdispersion}.

        Secondly, most of my dataset consists of zero
        observations. Although they do seem to be genuine observations
        and not caused by another proces (we will check for this
        later), they do need to be taken specifically into account. 

        I include 7 other variables in my model. First, 
        
        \section{Modeling framework}

        \subsection{The traditional gravity model}

        We adopt the basic gravity model specification pioneered by
        \citet{tinbergen1962shaping}.

        \citet{anderson2003gravity} argued that origin and destination
        specific variables should be incorporated to take into account
        multilateral resitance terms.

        Given that the variance is four times the mean of the
        migration flows, we need to correct for overdispersion of
        heterogeskedasticity \citep[][states that heteroskedasticity
        (rather than the presence of too many zeros) is responsible
        for the main differences.]{silva2006log}

        \begin{subequations}
          \begin{align} \text{Migrants}_{ij} \sim & \text{
              GammaPoisson}(\lambda_{ij}, \tau)\\ \log(\lambda_{ij}) =
            & \alpha + o_{\text{mun}[i]} + d_{\text{mun}[j]} + \notag
            \\ & \beta_1 \log(\text{home}_i) +
            \beta_2\log(\text{home}_j) + \notag \\ & \beta_3
            \log(\text{soc}_i) + \beta_4 \log(\text{soc}_j) + \notag\\
            & \beta_5 \log(\text{pop}_i) + \beta_6 \log(\text{pop}_j)
            + \notag \\ & \beta_7 \log(\text{dist}_{ij}) \\
            o_{\text{mun}} \sim& \text{ Normal}(\alpha_o, \sigma_o)\\
            d_{\text{mun}} \sim& \text{ Normal}(\alpha_d, \sigma_d)\\
            \beta_1,\ldots, \beta_7 \sim& \text{
                                          Normal}(0,2)\\ \alpha_o, \alpha_d \sim& \text{ Normal}(0,2)\\
            \sigma_o, \sigma_d \sim& \text{ HalfCauchy}(0,1) \\ \tau
            \sim& \text{ Gamma}(0.01, 0.01)
          \end{align}
          \label{model}
        \end{subequations}

        \section{Results}

        \subsection{Parameter estimates}
        
        I estimated model (\ref{model}) by using the \emph{No U-Turn
          Sampler} (NUTS) in Stan.\footnote{As interface to Stan
          \citep[see for an overview article of
          Stan][]{carpenter2017stan} I used the \texttt{brms}
          R-package \citep{brms}.} NUTS is a relatively recent
        developed Hamiltonian Monte Carlo (a specific form of Markov
        Chain Monte Carlo simulation) method, able to draw samples
        efficiently from large multilevel models
        \citep{hoffman2014no}.

% latex table generated in R 3.4.4 by xtable 1.8-3 package
% Fri Feb 22 15:07:02 2019
\begin{table}[ht]
  \centering
  \caption{Parameter estimates with 95\% probability intervals (group speficic origin and destination estimates are not presented)}
  \label{tab:coef}
  \begin{tabular}{lrrrr}
    \toprule
    Parameter & mean & sd & 2.5\% & 97.5\% \\ 
    \midrule
    b\_Intercept & -0.74 & 0.04 & -0.82 & -0.66 \\ 
    b\_pop\_d & 0.89 & 0.03 & 0.83 & 0.96 \\ 
    b\_pop\_o & 0.88 & 0.04 & 0.79 & 0.97 \\ 
    b\_hom\_d & -1.48 & 0.19 & -1.86 & -1.10 \\ 
    b\_hom\_o & -1.27 & 0.25 & -1.75 & -0.78 \\ 
    b\_soc\_o & -0.04 & 0.04 & -0.11 & 0.03 \\ 
    b\_soc\_d & -0.06 & 0.03 & -0.12 & -0.01 \\ 
    b\_log\_distance & -1.96 & 0.01 & -1.97 & -1.95 \\ 
    sd\_destination\_\_Intercept & 0.45 & 0.02 & 0.42 & 0.49 \\ 
    sd\_origin\_\_Intercept & 0.61 & 0.02 & 0.57 & 0.66 \\ 
    shape & 1.22 & 0.01 & 1.20 & 1.24 \\ 
    \bottomrule
\end{tabular}
\end{table}

\begin{figure}
  \includegraphics[width = \columnwidth]{../fig/forestplot.pdf}
  \caption{Forest plot of parameter means and 95\% probability intervals (group speficic origin and destination estimates are not presented)}
  \label{fig:forestplot}
\end{figure}

\subsection{Model predictions}

\section{In conclusion}
	
	%----------------------------------------------------------------------------------------
	%	REFERENCE LIST
	%----------------------------------------------------------------------------------------
	
	\addcontentsline{toc}{section}{references} % Adds this section to the table of contents
	\printbibliography
	
	%----------------------------------------------------------------------------------------
	
\end{document}
%%% Local Variables:
%%% mode: latex
%%% TeX-master: t
%%% End:
