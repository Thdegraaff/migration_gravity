\documentclass{beamer}
\usetheme[block = fill, titleformat title = allcaps, titleformat subtitle  = smallcaps,
progressbar = foot, background = light]{metropolis}           % Use metropolis theme
\usepackage{pgfplots}
\usepackage{appendixnumberbeamer}
\usepackage{booktabs}
\usepackage{amsmath, array}
\usetikzlibrary{arrows.meta,positioning, shapes, backgrounds}

\usepackage[sortcites=false,style=authoryear-comp,bibencoding=utf8, natbib=true, firstinits=true, maxcitenames=2, maxbibnames = 99, uniquename=false, backend=bibtex, useprefix=true, backref=false,doi=false,isbn=false,url=false,dashed=true]{biblatex}
\setlength\bibhang{20pt}
\bibliography{../../paper/references.bib}
\AtEveryBibitem{%
	\clearfield{day}%
	\clearfield{month}%
	\clearfield{endday}%
	\clearfield{endmonth}%
}
\AtBeginBibliography{\footnotesize}

\makeatletter
\let\save@measuring@true\measuring@true
\def\measuring@true{%
	\save@measuring@true
	\def\beamer@sortzero##1{\beamer@ifnextcharospec{\beamer@sortzeroread{##1}}{}}%
	\def\beamer@sortzeroread##1<##2>{}%
	\def\beamer@finalnospec{}%
}
\makeatother

\title{Housing market and migration revisited}
\subtitle{A multilevel gravity model for Dutch regions}
\date{August 25, 2020}
\author{Thomas de Graaff}
\institute{Vrije Universiteit Amsterdam\\Tinbergen Institute Amsterdam}
\begin{document}
\maketitle


\begin{frame}{Housing market and interregional migration: why bother?}
   	\begin{itemize}
   		\item Current Dutch housing market:
   		\begin{itemize}
   			\item large \alert{shortage} of housing (especially in popular, urban, regions and for private rental housing)
   			\item large yearly prices \alert{increases} (8\% in 2018)
   			\item decrease in number of houses \alert{sold}
   			\item large \alert{regional} variation\newline
   		\end{itemize}
                \item Policy debates about changes in housing market \alert{structure}
                \begin{itemize}
                \item $\longrightarrow$ Increase home-ownership rates\newline
                \end{itemize}
   		\item Large literature of \alert{external} effects of home-ownership \footnotesize{\citep{dietz2003social}}
   		\begin{itemize}
   			\item \alert{positive}: savings, labor supply, health, maintenance, etc.
   			\item \alert{negative}: migration (by increased moving costs) and on aggregate labour market performance \footnotesize{\citep{oswald1996conjecture,oswald1999housing}}
   		\end{itemize}
   	\end{itemize}
\end{frame}

\begin{frame}{My contributions to the literature}
  \begin{itemize}
  \item Large empirical (economic) literature on impact home-ownership on migration, but:
    \begin{itemize}
    \item usually concerns \alert{marginal} effect of home-ownership
    \item less attention to \alert{predictions} for the whole network\newline
    \end{itemize}
  \item Literature on impact of social renting on migration flows is
    scarce \footnotesize{\citep{de2009homeownership} }
	\begin{itemize}
        \item In the Netherlands social renting is a large phenomenon
          ($\approx$ 24\% of total housing stock)
        \item Social renting rights only valid \alert{within} city
        \item Social renting is an \alert{urban} phenomenon (e.g. $\approx$
          40--50\% in Amsterdam) 
        \end{itemize}
\end{itemize}
\end{frame}

\begin{frame}{So, this paper}
  \begin{description}
  \item[Does what?] \alert{Revisits} the impact of housing market
    structure (with focus on social renting) on within-country
    migration flows using a Bayesian multilevel gravity model
    \begin{footnotesize}
	\begin{itemize}
	\item \footnotesize UK context by \citet{congdon2010random}
	\item \footnotesize Social relations model \emph{cf.} \citet{koster2014food} \newline
\end{itemize}
    \end{footnotesize}
  \item[And] \alert{Predict} all changes in incoming and
    outcoming migration flows when housing market structure changes
\end{description}
\end{frame}


\begin{frame}[fragile]{Why a \textbf{multilevel} approach for the gravity model}
\begin{figure}	
	\begin{tikzpicture}[scale=0.90, thick]
	
	\tikzstyle{orig}=[rectangle, rounded corners, thin, fill=green!20, text=black, draw, minimum width=2.5cm]
	\tikzstyle{dest}=[rectangle, rounded corners, thin, fill=red!20, text=black, draw, minimum width=2.5cm]
	\tikzstyle{migrants}=[rectangle, rounded corners, thin, fill=blue!20, text=black, draw, minimum width=2.5cm]
	\tikzstyle{var}=[rectangle, rounded corners, thin, fill=black!10, text=black, draw, text width = 3.5cm]
	
	\node[orig] (o1) at (0,0)  {\textsc{City$_1$}};
	\node[orig] (o2) at (0,-1) {\textsc{City$_2$}};
	\node[orig] (o3) at (0,-2) {\textsc{City$_3$}};
	\node (o4) at (0,-3) {\textsc{\vdots}};
	\node[orig] (o5) at (0,-4) {\textsc{City$_i$}};
	\node (o6) at (0,-5) {\textsc{\vdots}};
	\node[orig] (o7) at (0,-6) {\textsc{City$_I$}};
	
	\end{tikzpicture}
\end{figure}
\end{frame}

\begin{frame}[fragile]{Why a \textbf{multilevel} approach for the gravity model}
	 \begin{figure}	
   \begin{tikzpicture}[scale=0.90, thick]
   
   \tikzstyle{orig}=[rectangle, rounded corners, thin, fill=green!20, text=black, draw, minimum width=2.5cm]
   \tikzstyle{dest}=[rectangle, rounded corners, thin, fill=red!20, text=black, draw, minimum width=2.5cm]
   \tikzstyle{migrants}=[rectangle, rounded corners, thin, fill=blue!20, text=black, draw, minimum width=2.5cm]
   \tikzstyle{var}=[rectangle, rounded corners, thin, fill=black!10, text=black, draw, minimum width = 2.5cm]
   
   \node[orig] (o1) at (0,0)  {\textsc{Origin$_1$}};
   \node[orig] (o2) at (0,-1) {\textsc{Origin$_2$}};
   \node[orig] (o3) at (0,-2) {\textsc{Origin$_3$}};
   \node (o4) at (0,-3) {\textsc{\vdots}};
   \node[orig] (o5) at (0,-4) {\textsc{Origin$_i$}};
   \node (o6) at (0,-5) {\textsc{\vdots}};
   \node[orig] (o7) at (0,-6) {\textsc{Origin$_I$}};
   
   \node[dest] (d1) at (8,0)  {\textsc{Destination$_1$}};
   \node[dest] (d2) at (8,-1) {\textsc{Destination$_2$}};
   \node[dest] (d3) at (8,-2) {\textsc{Destination$_3$}};
   \node (d4) 		 at (8,-3) {\textsc{\vdots}};
   \node[dest] (d5) at (8,-4) {\textsc{Destination$_j$}};
   \node (d6) 		 at (8,-5) {\textsc{\vdots}};
   \node[dest] (d7) at (8,-6) {\textsc{Destination$_J$}};
%   
%   % Migratior links
%   \draw[-latex, blue, thin, dashed] (1.5,0) -- (6.5, 0);
%   \draw[-latex, blue, thin, dashed] (1.5,0) -- (6.5,-0.8);
%   \draw[-latex, blue, thin, dashed] (1.5,0) -- (6.5,-1.8);
%   \draw[-latex, blue, thin, dashed] (1.5,0) -- (6.5,-3.8);
%   \draw[-latex, blue, thin, dashed] (1.5,0) -- (6.5,-5.6);
%   
%   \draw[-latex, blue, thin, dashed] (1.5,-1) -- (6.5,-0.1);
%   \draw[-latex, blue, thin, dashed] (1.5,-1) -- (6.5,-0.9);
%   \draw[-latex, blue, thin, dashed] (1.5,-1) -- (6.5,-1.9);
%   \draw[-latex, blue, thin, dashed] (1.5,-1) -- (6.5,-3.9);
%   \draw[-latex, blue, thin, dashed] (1.5,-1) -- (6.5,-5.7);
%   
%   \draw[-latex, blue, thin, dashed] (1.5,-2) -- (6.5, -0.2);
%   \draw[-latex, blue, thin, dashed] (1.5,-2) -- (6.5,-1);
%   \draw[-latex, blue, thin, dashed] (1.5,-2) -- (6.5,-2);
%   \draw[-latex, blue, thin, dashed] (1.5,-2) -- (6.5,-4);
%   \draw[-latex, blue, thin, dashed] (1.5,-2) -- (6.5,-5.8);
%   
%   \draw[-latex, blue, thin, dashed] (1.5,-4) -- (6.5,-0.3);
%   \draw[-latex, blue, thin, dashed] (1.5,-4) -- (6.5,-1.1);
%   \draw[-latex, blue, thin, dashed] (1.5,-4) -- (6.5,-2.1);
%   \draw[-latex, blue, thin, dashed] (1.5,-4) -- (6.5,-4.1);
%   \draw[-latex, blue, thin, dashed] (1.5,-4) -- (6.5,-5.9);
%   
%   \draw[-latex, blue, thin, dashed] (1.5,-6) -- (6.5, -0.4);
%   \draw[-latex, blue, thin, dashed] (1.5,-6) -- (6.5,-1.2);
%   \draw[-latex, blue, thin, dashed] (1.5,-6) -- (6.5,-2.2);
%   \draw[-latex, blue, thin, dashed] (1.5,-6) -- (6.5,-4.2);
%   \draw[-latex, blue, thin, dashed] (1.5,-6) -- (6.5,-6);
%   
%   \node[migrants] (m) at (4,-3)  {\textsc{Flows $i \rightarrow j$}};
%   
   \node[var] (v1) at (0,-8)  { \textsc{Push factors} };
%   \node[var] (v2) at (4,-8)  {\textsc{flow-specific variables ($\mathbf{X}_{ij}$) }};
   \node[var] (v3) at (8,-8)  { \textsc{Pull factors} };
%   
%   \begin{pgfonlayer}{background}
%   \filldraw [line width=4mm,join=round,black!10]
%   (o1.north -| o1.west)  rectangle (o7.south -| o7.east)
%   (d1.north -| d1.west)  rectangle (d7.south -| d7.east);
%   \end{pgfonlayer}
%   
   \draw[-latex, thick, black] (v1) -- (0,-6.5);
%   \draw[-latex, thick, black] (v2) -- (4,-6.5);
   \draw[-latex, thick, black] (v3) -- (8,-6.5);
   \end{tikzpicture}
	\end{figure}
\end{frame}

\begin{frame}[fragile]{Why a \textbf{multilevel} approach structure for the gravity model}
	\begin{figure}	
		\begin{tikzpicture}[scale=0.90, thick]
		
		\tikzstyle{orig}=[rectangle, rounded corners, thin, fill=green!20, text=black, draw, minimum width=2.5cm]
		\tikzstyle{dest}=[rectangle, rounded corners, thin, fill=red!20, text=black, draw, minimum width=2.5cm]
		\tikzstyle{migrants}=[rectangle, rounded corners, thin, fill=blue!20, text=black, draw, minimum width=2.5cm]
		\tikzstyle{var}=[rectangle, rounded corners, thin, fill=black!10, text=black, draw, minimum width = 2.5cm]
		
		\node[orig] (o1) at (0,0)  {\textsc{Origin$_1$}};
		\node[orig] (o2) at (0,-1) {\textsc{Origin$_2$}};
		\node[orig] (o3) at (0,-2) {\textsc{Origin$_3$}};
		\node (o4) at (0,-3) {\textsc{\vdots}};
		\node[orig] (o5) at (0,-4) {\textsc{Origin$_i$}};
		\node (o6) at (0,-5) {\textsc{\vdots}};
		\node[orig] (o7) at (0,-6) {\textsc{Origin$_I$}};
		
		\node[dest] (d1) at (8,0)  {\textsc{Destination$_1$}};
		\node[dest] (d2) at (8,-1) {\textsc{Destination$_2$}};
		\node[dest] (d3) at (8,-2) {\textsc{Destination$_3$}};
		\node (d4) 		 at (8,-3) {\textsc{\vdots}};
		\node[dest] (d5) at (8,-4) {\textsc{Destination$_j$}};
		\node (d6) 		 at (8,-5) {\textsc{\vdots}};
		\node[dest] (d7) at (8,-6) {\textsc{Destination$_J$}};
		   
		   % Migratior links
		   \draw[-latex, blue, thin, dashed] (1.5,0) -- (6.5, 0);
		   \draw[-latex, blue, thin, dashed] (1.5,0) -- (6.5,-0.8);
		   \draw[-latex, blue, thin, dashed] (1.5,0) -- (6.5,-1.8);
		   \draw[-latex, blue, thin, dashed] (1.5,0) -- (6.5,-3.8);
		   \draw[-latex, blue, thin, dashed] (1.5,0) -- (6.5,-5.6);
		   
		   \draw[-latex, blue, thin, dashed] (1.5,-1) -- (6.5,-0.1);
		   \draw[-latex, blue, thin, dashed] (1.5,-1) -- (6.5,-0.9);
		   \draw[-latex, blue, thin, dashed] (1.5,-1) -- (6.5,-1.9);
		   \draw[-latex, blue, thin, dashed] (1.5,-1) -- (6.5,-3.9);
		   \draw[-latex, blue, thin, dashed] (1.5,-1) -- (6.5,-5.7);
		   
		   \draw[-latex, blue, thin, dashed] (1.5,-2) -- (6.5, -0.2);
		   \draw[-latex, blue, thin, dashed] (1.5,-2) -- (6.5,-1);
		   \draw[-latex, blue, thin, dashed] (1.5,-2) -- (6.5,-2);
		   \draw[-latex, blue, thin, dashed] (1.5,-2) -- (6.5,-4);
		   \draw[-latex, blue, thin, dashed] (1.5,-2) -- (6.5,-5.8);
		   
		   \draw[-latex, blue, thin, dashed] (1.5,-4) -- (6.5,-0.3);
		   \draw[-latex, blue, thin, dashed] (1.5,-4) -- (6.5,-1.1);
		   \draw[-latex, blue, thin, dashed] (1.5,-4) -- (6.5,-2.1);
		   \draw[-latex, blue, thin, dashed] (1.5,-4) -- (6.5,-4.1);
		   \draw[-latex, blue, thin, dashed] (1.5,-4) -- (6.5,-5.9);
		   
		   \draw[-latex, blue, thin, dashed] (1.5,-6) -- (6.5, -0.4);
		   \draw[-latex, blue, thin, dashed] (1.5,-6) -- (6.5,-1.2);
		   \draw[-latex, blue, thin, dashed] (1.5,-6) -- (6.5,-2.2);
		   \draw[-latex, blue, thin, dashed] (1.5,-6) -- (6.5,-4.2);
		   \draw[-latex, blue, thin, dashed] (1.5,-6) -- (6.5,-6);
		   
		   \node[migrants] (m) at (4,-3)  {\textsc{Flows $i \rightarrow j$}};
		   
		   \node[var] (v1) at (0,-8)  {\textsc{Push factors}};
		   \node[var] (v2) at (4,-8)  {\textsc{Frictions }};
		   \node[var] (v3) at (8,-8)  {\textsc{Pull factors} };
		%   
		%   \begin{pgfonlayer}{background}
		%   \filldraw [line width=4mm,join=round,black!10]
		%   (o1.north -| o1.west)  rectangle (o7.south -| o7.east)
		%   (d1.north -| d1.west)  rectangle (d7.south -| d7.east);
		%   \end{pgfonlayer}
		%   
		   \draw[-latex, thick, black] (v1) -- (0,-6.5);
		   \draw[-latex, thick, black] (v2) -- (4,-6.5);
		   \draw[-latex, thick, black] (v3) -- (8,-6.5);
		\end{tikzpicture}
	\end{figure}
\end{frame}

\begin{frame}[fragile]{Why a \textbf{multilevel} approach for the gravity model}
	\begin{figure}	
		\begin{tikzpicture}[scale=0.90, thick]
		
		\tikzstyle{orig}=[rectangle, rounded corners, thin, fill=green!20, text=black, draw, minimum width=2.5cm]
		\tikzstyle{dest}=[rectangle, rounded corners, thin, fill=red!20, text=black, draw, minimum width=2.5cm]
		\tikzstyle{migrants}=[rectangle, rounded corners, thin, fill=blue!20, text=black, draw, minimum width=2.5cm]
		\tikzstyle{var}=[rectangle, rounded corners, thin, fill=black!10, text=black, draw, minimum width = 2.5cm]
		
		\node[orig] (o1) at (0,0)  {\textsc{Origin$_1$}};
		\node[orig] (o2) at (0,-1) {\textsc{Origin$_2$}};
		\node[orig] (o3) at (0,-2) {\textsc{Origin$_3$}};
		\node (o4) at (0,-3) {\textsc{\vdots}};
		\node[orig] (o5) at (0,-4) {\textsc{Origin$_i$}};
		\node (o6) at (0,-5) {\textsc{\vdots}};
		\node[orig] (o7) at (0,-6) {\textsc{Origin$_I$}};
		
		\node[dest] (d1) at (8,0)  {\textsc{Destination$_1$}};
		\node[dest] (d2) at (8,-1) {\textsc{Destination$_2$}};
		\node[dest] (d3) at (8,-2) {\textsc{Destination$_3$}};
		\node (d4) 		 at (8,-3) {\textsc{\vdots}};
		\node[dest] (d5) at (8,-4) {\textsc{Destination$_j$}};
		\node (d6) 		 at (8,-5) {\textsc{\vdots}};
		\node[dest] (d7) at (8,-6) {\textsc{Destination$_J$}};
		
		% Migratior links
		\draw[-latex, blue, thin, dashed] (1.5,0) -- (6.5, 0);
		\draw[-latex, blue, thin, dashed] (1.5,0) -- (6.5,-0.8);
		\draw[-latex, blue, thin, dashed] (1.5,0) -- (6.5,-1.8);
		\draw[-latex, blue, thin, dashed] (1.5,0) -- (6.5,-3.8);
		\draw[-latex, blue, thin, dashed] (1.5,0) -- (6.5,-5.6);
		
		\draw[-latex, blue, thin, dashed] (1.5,-1) -- (6.5,-0.1);
		\draw[-latex, blue, thin, dashed] (1.5,-1) -- (6.5,-0.9);
		\draw[-latex, blue, thin, dashed] (1.5,-1) -- (6.5,-1.9);
		\draw[-latex, blue, thin, dashed] (1.5,-1) -- (6.5,-3.9);
		\draw[-latex, blue, thin, dashed] (1.5,-1) -- (6.5,-5.7);
		
		\draw[-latex, blue, thin, dashed] (1.5,-2) -- (6.5, -0.2);
		\draw[-latex, blue, thin, dashed] (1.5,-2) -- (6.5,-1);
		\draw[-latex, blue, thin, dashed] (1.5,-2) -- (6.5,-2);
		\draw[-latex, blue, thin, dashed] (1.5,-2) -- (6.5,-4);
		\draw[-latex, blue, thin, dashed] (1.5,-2) -- (6.5,-5.8);
		
		\draw[-latex, blue, thin, dashed] (1.5,-4) -- (6.5,-0.3);
		\draw[-latex, blue, thin, dashed] (1.5,-4) -- (6.5,-1.1);
		\draw[-latex, blue, thin, dashed] (1.5,-4) -- (6.5,-2.1);
		\draw[-latex, blue, thin, dashed] (1.5,-4) -- (6.5,-4.1);
		\draw[-latex, blue, thin, dashed] (1.5,-4) -- (6.5,-5.9);
		
		\draw[-latex, blue, thin, dashed] (1.5,-6) -- (6.5, -0.4);
		\draw[-latex, blue, thin, dashed] (1.5,-6) -- (6.5,-1.2);
		\draw[-latex, blue, thin, dashed] (1.5,-6) -- (6.5,-2.2);
		\draw[-latex, blue, thin, dashed] (1.5,-6) -- (6.5,-4.2);
		\draw[-latex, blue, thin, dashed] (1.5,-6) -- (6.5,-6);
		
		\node[migrants] (m) at (4,-3)  {\textsc{Flows $i \rightarrow j$}};
		
		\node[var] (v1) at (0,-8)  {\textsc{Origin  ($o_i, \mathbf{X_i}$)}};
		\node[var] (v2) at (4,-8)  {\textsc{Flow ($\mathbf{X}_{ij}$) }};
		   \node[var] (v3) at (8,-8)  {\textsc{Destination ($d_j, \mathbf{X_j}$)}};
		   
%		   \begin{pgfonlayer}{background}
%		   \filldraw [line width=4mm,join=round,black!10]
%		   (o1.north -| o1.west)  rectangle (o7.south -| o7.east)
%		   (d1.north -| d1.west)  rectangle (d7.south -| d7.east);
%		   \end{pgfonlayer}
		   
		   \draw[-latex, thick, black] (v1) -- (0,-6.5);
		\draw[-latex, thick, black] (v2) -- (4,-6.5);
		   \draw[-latex, thick, black] (v3) -- (8,-6.5);
		\end{tikzpicture}
	\end{figure}
\end{frame}

\begin{frame}{Why a Bayesian multilevel approach?}
\begin{itemize}
	\item Hierarchical, mixed effects, varying intercept/parameter, shrinkage, partial pooling models\pause
	\item Increasingly used for model \alert{performance} and \alert{flexibility} \pause
    \item \alert{Simultaneous} modeling at various levels (e.g., cities, regions, flows, individuals) 
    \begin{itemize}
    	\item no two-stage models anymore 
    	\item precision (standard errors) is correct\pause
    \end{itemize}
	\item \alert{Partial pooling}: origin and destination specific effects are draws from a distribution: usually $\sim \text{ Normal}(\alpha, \sigma)$
	\begin{itemize}
		\item $\sigma \longrightarrow 0$ : complete pooling
		\item $\sigma \longrightarrow \infty$ : no pooling (fixed effects)
	\end{itemize}
\end{itemize}
\end{frame}

\begin{frame}{Data: migrations flows}
	\begin{center}
		\includegraphics[width=0.8\textwidth]{../../fig/hist_mig}      
	\end{center}
\begin{itemize}
	\item Migration flows \alert{between} 380 Dutch municipalities in 2018 ($\approx 144,000$ flows)
	\item Variance is 4 times expectation: \alert{over-dispersion}
\end{itemize}
\end{frame}

\begin{frame}{Data: municipal housing structure}
\begin{center}
	\includegraphics[width=0.8\textwidth]{../../fig/hist_housing}      
\end{center}
\begin{itemize}
	\item Positive correlation between city size and social renting ($0.4$)
	\item Negative correlation between social renting and home-ownership ($-0.84)$
\end{itemize}
\end{frame}

\begin{frame}{Modeling framework: traditional gravity modeling}
	\begin{equation*}
	\log(\text{Migrants}_{ij}) = o_i + d_j + \gamma\log(\text{dist}_{ij}) + \epsilon_{ij}
	\label{eq:gravfixed}
	\end{equation*} 
	
	Origin and destination specific \alert{fixed} effects for multilateral resistance  \citep{anderson2003gravity} , but:
	\begin{itemize}
		\item what about \alert{zeros} in $\text{Migrants}_{ij}$?
		\item how to incorporate \alert{housing} structure in the presence of $o_i$ and $d_j$?
		\item \alert{over-dispersion} and \alert{heteroskedasticity} \footnotesize{\citep{silva2006log} }
	\end{itemize}
\end{frame}

\begin{frame}{Poisson versus negative binomial\footnote{
			\emph{We urge researchers to resist the siren song of the Negative
			Binomial \footnotesize{ \citep{head2014gravity}} }
			}}
	
	\begin{itemize}
		\item Counts of migrants\newline 
		\item 	With Poisson \& regional effects of origin and destination the following origin and destination \alert{constraints} automatically hold
		$$
		\sum_{j=1}^{R} {\widehat{\text{Migrants} }_{ij} } = O_i \qquad \sum_{i=1}^{R} {\widehat{\text{Migrants} }_{ij} } = D_j
		$$

	\item  Does not apply with negative binomial \newline
	\item Multilevel model accounts for dispersion
		\end{itemize}

\end{frame}

\begin{frame}[fragile]{Modeling framework: multilevel gravity modeling}
\begin{small}
	\begin{align} \text{Migrants}_{ij} \sim & \text{ Poisson}(\lambda_{ij}) \tag{\footnotesize \color{blue} flow of migrants} \\ \pause
	\log(\lambda_{ij}) =
	& \: \alpha + o_{\text{mun}[i]} + d_{\text{mun}[j]} + \tag{\footnotesize
   \color{blue} city effects}  \\ \pause
	& \: \beta_1 \log(\text{pop}_i) + \notag
	\beta_2\log(\text{pop}_j) + \notag \\ & \beta_3
	\log(\text{home}_i) + \beta_4 \log(\text{home}_j) + \notag\\
	& \: \beta_5 \log(\text{soc}_i) + \beta_6 \log(\text{soc}_j) + \notag \\ 
	& \: \beta_7 \log(\text{dist}_{ij})  \tag{\footnotesize \color{blue} explanatory variables}  \\ \pause
	\left(\begin{matrix} 
	o_{\text{mun}} \\
	d_{\text{mun}}
	\end{matrix}\right) \sim& \text{ MVNormal}\left( \left( \begin{matrix}
	0 \\
	0 
	\end{matrix}\right) , \left(\begin{matrix}
	\sigma_o^2 & \sigma_o \sigma_d \rho \\
	 \sigma_o \sigma_d \rho  & \sigma_d^2
	\end{matrix}\right) \right)  \tag{\footnotesize \color{blue} varying region effects}  \\ \pause
			\beta_7 \sim& \text{ Normal}(-1,1) \tag{\footnotesize \color{blue} prior} \\ 
	\alpha, \beta_1,\ldots, \beta_6 \sim& \text{
		Normal}(0,2) \tag{\footnotesize \color{blue} priors} \\ 
	\sigma_o, \sigma_d \sim& \text{ HalfCauchy}(0,1) \tag{\footnotesize \color{blue} priors} 
	\end{align}
\end{small}
\end{frame}





\begin{frame}{Estimation results}
		\begin{columns}
		\begin{column}{0.5\textwidth}
			\begin{center}
				\begin{small}
			  \begin{tabular}{lrr}
				\toprule
				Parameter & mean & sd \\ 
				\midrule
				Intercept      & $-0.34$ & 0.05 \\ 
				log(pop$_i$)   & 0.88 & 0.04  \\ 
				log(pop$_j$)   & 0.91 & 0.05  \\ 
				log(home$_i$)  & $-0.44$ & 0.18 \\ 
				log(home$_j$)  & $-0.69$ & 0.22  \\ 
				log(soc$_i$)   & $-0.08$ & 0.03  \\
				log(soc$_j$)   & $-0.08$ & 0.04 \\ 
				log(dist$_{ij})$ & $-1.88$ & 0.01  \\ 
				$\sigma_o$    & 0.54 & 0.02  \\ 
				$\sigma_d$    & 0.43 & 0.02  \\ 
				$\rho$        & 0.88 & 0.01  \\ 
				\bottomrule
			\end{tabular}
		\end{small}
		\end{center}
		\end{column}
		\begin{column}{0.5\textwidth} 

			\begin{center}
				\includegraphics[width=\textwidth]{../../fig/forestplot}      
			\end{center}
		\end{column}
	\end{columns}
\end{frame}

\begin{frame}{Robustness checks for impact social renting}
			\begin{columns}
		\begin{column}{0.5\textwidth}
			\begin{itemize}
				\item  \alert{impact} $soc \approx 0$ \newline \pause
				\item Robust to:
				\begin{itemize}
					\item year
					\item interaction effects with population \& home-ownership
					\item varying slopes (e.g, $o_{\text{mun}[i]} + \beta_{\text{mun}[i]} \log(\text{soc}_i)$) 
					\item Inclusion of household size\pause
				\end{itemize}
			\end{itemize}
		\end{column}
		\begin{column}{0.5\textwidth} 
			\alert{Assuming} the following causal graph (\emph{dag} )
			\begin{center}
				\includegraphics[width=\textwidth]{../../fig/dag}      
			\end{center}
		\end{column}
	\end{columns}
\end{frame}

\begin{frame}{Observed versus predicted flows}
\begin{center}
	\includegraphics[width=0.9\textwidth]{../../fig/hist_fit}      
\end{center}
\begin{itemize}
	\item \alert{maximum} observed flow: 3,573
	\item \alert{maximum} predicted flow: 3,584
\end{itemize}
\end{frame}

\begin{frame}{Correlation between origin an destination $\rho = 0.88$}
				\begin{center}
		\includegraphics[width=\textwidth]{../../fig/correlation}      
	\end{center}
\end{frame}


\begin{frame}{$d_{\text{mun}}$ as measure of attractivity?}
		\begin{columns}
	\begin{column}{0.5\textwidth}
		\begin{center}
			\includegraphics[width=1.1\textwidth]{../../fig/p_coef_out}      
		\end{center}
	\end{column}
	\begin{column}{0.5\textwidth} 	
		\begin{center}
			\includegraphics[width=1.1\textwidth]{../../fig/p_coef_in}      
		\end{center}
	\end{column}
\end{columns}
\end{frame}

\begin{frame}{Home-ownership rate in Amsterdam increases with 10\% and social renting decreases with 10\%}
	\begin{columns}
		\begin{column}{0.5\textwidth}
			\begin{center}
				\includegraphics[width=1.1\textwidth]{../../fig/p_diff_out}      
			\end{center}
		\end{column}
		\begin{column}{0.5\textwidth} 	
			\begin{center}
				\includegraphics[width=1.1\textwidth]{../../fig/p_diff_in}      
			\end{center}
		\end{column}
	\end{columns}
\end{frame}

\begin{frame}{Conclusions}
\textbf{Flexibel and powerful Bayesian multilevel gravity model}:
\begin{itemize}
	\item homeownership has a a high negative impact on migration
	\item social renting has elasticity close to \alert{zero} \citep{boyle1998migration}
	\begin{itemize}
		\item social housing is almost \alert{a different ball game}
		\item no interaction with homeowership, private renting or household size
	\end{itemize}
\end{itemize}

\textbf{Now what?}
\begin{itemize}
	\item additional model structure
	\begin{itemize}
		\item correlation between \alert{dyads} ($\text{Migrants}_{ij}$ and $\text{Migrants}_{ji}$)
		\item \alert{spatial} autocorrelation in regional varying effects
		\item create panel data
	\end{itemize}
\item Additional covariates (age, income)
\end{itemize}
\end{frame}

\begin{frame}{Supplementary materials}

Paper, presentation, data and code can be retrieved from the project's GitHub page: 

\begin{center}\url{https://github.com/Thdegraaff/migration\_gravity}\end{center}

\end{frame}

\begin{frame}[standout]
Thank you!
\end{frame}

\appendix

\begin{frame}[allowframebreaks]{References}

		\printbibliography[heading=none]

\end{frame}


\end{document}